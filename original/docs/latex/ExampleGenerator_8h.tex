\section{Example\-Generator.h File Reference}
\label{ExampleGenerator_8h}\index{ExampleGenerator.h@{ExampleGenerator.h}}


\subsection{Detailed Description}
Generate a random (but reproducible) data set. 

Randomly, but reproducably, create a series of examples. These examples could then be classified with some known model and used as a synthetic dataset to test a learner. This uses the {\bf Random\-Set\-State()} functions so that it will produce the same series of examples for the same seed no matter what the rest of your program does with the random number generators.

\subsection*{Data Structures}
\begin{CompactItemize}
\item 
struct {\bf \_\-Example\-Generator\_\-}
\begin{CompactList}\small\item\em Holds the information needed to reproducibly make a random data set. See {\bf Example\-Generator.h} for more detail. \item\end{CompactList}\end{CompactItemize}
\subsection*{Typedefs}
\begin{CompactItemize}
\item 
typedef {\bf \_\-Example\-Generator\_\-} {\bf Example\-Generator}
\begin{CompactList}\small\item\em Holds the information needed to reproducibly make a random data set. See {\bf Example\-Generator.h} for more detail. \item\end{CompactList}\item 
typedef {\bf \_\-Example\-Generator\_\-} $\ast$ {\bf Example\-Generator\-Ptr}
\begin{CompactList}\small\item\em Holds the information needed to reproducibly make a random data set. See {\bf Example\-Generator.h} for more detail. \item\end{CompactList}\end{CompactItemize}
\subsection*{Functions}
\begin{CompactItemize}
\item 
{\bf Example\-Generator\-Ptr} {\bf Example\-Generator\-New} ({\bf Example\-Spec\-Ptr} es, int seed)
\begin{CompactList}\small\item\em Creates a new example generator. \item\end{CompactList}\item 
void {\bf Example\-Generator\-Free} ({\bf Example\-Generator\-Ptr} eg)
\begin{CompactList}\small\item\em Frees the memory associated with the example generator. \item\end{CompactList}\item 
{\bf Example\-Ptr} {\bf Example\-Generator\-Generate} ({\bf Example\-Generator\-Ptr} eg)
\begin{CompactList}\small\item\em Makes a random example. \item\end{CompactList}\end{CompactItemize}


\subsection{Typedef Documentation}
\index{ExampleGenerator.h@{Example\-Generator.h}!ExampleGenerator@{ExampleGenerator}}
\index{ExampleGenerator@{ExampleGenerator}!ExampleGenerator.h@{Example\-Generator.h}}
\subsubsection{\setlength{\rightskip}{0pt plus 5cm}typedef struct {\bf \_\-Example\-Generator\_\-}  {\bf Example\-Generator}}\label{ExampleGenerator_8h_a0}


Holds the information needed to reproducibly make a random data set. See {\bf Example\-Generator.h} for more detail. 

\index{ExampleGenerator.h@{Example\-Generator.h}!ExampleGeneratorPtr@{ExampleGeneratorPtr}}
\index{ExampleGeneratorPtr@{ExampleGeneratorPtr}!ExampleGenerator.h@{Example\-Generator.h}}
\subsubsection{\setlength{\rightskip}{0pt plus 5cm}typedef struct {\bf \_\-Example\-Generator\_\-} $\ast$ {\bf Example\-Generator\-Ptr}}\label{ExampleGenerator_8h_a1}


Holds the information needed to reproducibly make a random data set. See {\bf Example\-Generator.h} for more detail. 



\subsection{Function Documentation}
\index{ExampleGenerator.h@{Example\-Generator.h}!ExampleGeneratorFree@{ExampleGeneratorFree}}
\index{ExampleGeneratorFree@{ExampleGeneratorFree}!ExampleGenerator.h@{Example\-Generator.h}}
\subsubsection{\setlength{\rightskip}{0pt plus 5cm}void Example\-Generator\-Free ({\bf Example\-Generator\-Ptr} {\em eg})}\label{ExampleGenerator_8h_a3}


Frees the memory associated with the example generator. 

\index{ExampleGenerator.h@{Example\-Generator.h}!ExampleGeneratorGenerate@{ExampleGeneratorGenerate}}
\index{ExampleGeneratorGenerate@{ExampleGeneratorGenerate}!ExampleGenerator.h@{Example\-Generator.h}}
\subsubsection{\setlength{\rightskip}{0pt plus 5cm}{\bf Example\-Ptr} Example\-Generator\-Generate ({\bf Example\-Generator\-Ptr} {\em eg})}\label{ExampleGenerator_8h_a4}


Makes a random example. 

Allocates an example, randomly sets the values of its attributes, and returns it. Uses uniform distributions for all of its decisions. For continuous attributes it uniformly generates a value between 0 and 1.0; you might like to scale this value to fit your needs.

You are must free the Example\-Ptr when you are finished with it by calling Example\-Free. \index{ExampleGenerator.h@{Example\-Generator.h}!ExampleGeneratorNew@{ExampleGeneratorNew}}
\index{ExampleGeneratorNew@{ExampleGeneratorNew}!ExampleGenerator.h@{Example\-Generator.h}}
\subsubsection{\setlength{\rightskip}{0pt plus 5cm}{\bf Example\-Generator\-Ptr} Example\-Generator\-New ({\bf Example\-Spec\-Ptr} {\em es}, int {\em seed})}\label{ExampleGenerator_8h_a2}


Creates a new example generator. 

The generator will generate examples conforming to es using a seeded random number generator. If the value of seed is -1 this will select a random initial seed (But you'll need to initialize the random number generator on your system for this to work; The function {\bf Random\-Init()} will do the job). 