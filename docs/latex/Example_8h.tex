\section{Example.h File Reference}
\label{Example_8h}\index{Example.h@{Example.h}}


\subsection{Detailed Description}
ADT for training (and testing, etc.) data. 

This is the interface for working with instances of the Example ADT. Note that all access to attributes is 0 based (just like C arrays).

Note that all the Examples created with an Example\-Spec maintain a pointer to the Example\-Spec, so you shouldn't free it or modify the Example\-Spec until you are done with all the Examples referencing it.

\subsection*{Data Structures}
\begin{CompactItemize}
\item 
struct {\bf \_\-Example\_\-}
\begin{CompactList}\small\item\em ADT for working with examples. \item\end{CompactList}\end{CompactItemize}
\subsection*{Defines}
\begin{CompactItemize}
\item 
\#define {\bf Example\-Is\-Attribute\-Unknown}(e, att\-Num)\ ( VALIndex(e $\rightarrow$ attributes, att\-Num) == 0 )
\begin{CompactList}\small\item\em Returns 1 if the specified attribute is marked as unknown and 0 otherwise. \item\end{CompactList}\item 
\#define {\bf Example\-Get\-Num\-Attributes}(e)\ (VALLength((({\bf Example\-Ptr})e) $\rightarrow$ attributes))
\begin{CompactList}\small\item\em Returns the number of attributes that this example has. \item\end{CompactList}\item 
\#define {\bf Example\-Get\-Class}(e)\ ((({\bf Example\-Ptr})e) $\rightarrow$ myclass)
\begin{CompactList}\small\item\em Returns the value of the example's class. \item\end{CompactList}\end{CompactItemize}
\subsection*{Typedefs}
\begin{CompactItemize}
\item 
typedef {\bf \_\-Example\_\-} {\bf Example}
\begin{CompactList}\small\item\em ADT for working with examples. \item\end{CompactList}\item 
typedef {\bf \_\-Example\_\-} $\ast$ {\bf Example\-Ptr}
\begin{CompactList}\small\item\em ADT for working with examples. \item\end{CompactList}\end{CompactItemize}
\subsection*{Functions}
\begin{CompactItemize}
\item 
{\bf Example\-Ptr} {\bf Example\-New} ({\bf Example\-Spec\-Ptr} es)
\begin{CompactList}\small\item\em Programmatically creates a new example. \item\end{CompactList}\item 
void {\bf Example\-Free} ({\bf Example\-Ptr} e)
\begin{CompactList}\small\item\em Frees all the memory being used by the passed example. \item\end{CompactList}\item 
{\bf Example\-Ptr} {\bf Example\-Clone} ({\bf Example\-Ptr} e)
\begin{CompactList}\small\item\em Allocates memory and copies the example into it. \item\end{CompactList}\item 
void {\bf Example\-Set\-Attribute\-Unknown} ({\bf Example\-Ptr} e, int att\-Num)
\begin{CompactList}\small\item\em Marks the specified attribute value as unknown. \item\end{CompactList}\item 
void {\bf Example\-Set\-Discrete\-Attribute\-Value} ({\bf Example\-Ptr} e, int att\-Num, int value)
\begin{CompactList}\small\item\em Considers the specified attribute to be discrete and sets its value to the specified value. \item\end{CompactList}\item 
void {\bf Example\-Set\-Continuous\-Attribute\-Value} ({\bf Example\-Ptr} e, int att\-Num, double value)
\begin{CompactList}\small\item\em Considers the specified attribute to be continuous and sets its value to the specified value. \item\end{CompactList}\item 
void {\bf Example\-Set\-Class} ({\bf Example\-Ptr} e, int the\-Class)
\begin{CompactList}\small\item\em Sets the example's class to the specified class. \item\end{CompactList}\item 
void {\bf Example\-Set\-Class\-Unknown} ({\bf Example\-Ptr} e)
\begin{CompactList}\small\item\em Marks the example's class as unknown. \item\end{CompactList}\item 
void {\bf Example\-Add\-Noise} ({\bf Example\-Ptr} e, float p, int do\-Class, int attrib)
\begin{CompactList}\small\item\em Randomly corrupts the attributes and class of the example. \item\end{CompactList}\item 
{\bf Example\-Ptr} {\bf Example\-Read} (FILE $\ast$file, {\bf Example\-Spec\-Ptr} es)
\begin{CompactList}\small\item\em Attempts to read an example from the passed file pointer. \item\end{CompactList}\item 
{\bf Void\-List\-Ptr} {\bf Examples\-Read} (FILE $\ast$file, {\bf Example\-Spec\-Ptr} es)
\begin{CompactList}\small\item\em Reads as many examples as possible from the file pointer. \item\end{CompactList}\item 
int {\bf Example\-Is\-Attribute\-Discrete} ({\bf Example\-Ptr} e, int att\-Num)
\begin{CompactList}\small\item\em Returns 1 if the specified attribute is discrete and 0 otherwise. \item\end{CompactList}\item 
int {\bf Example\-Is\-Attribute\-Continuous} ({\bf Example\-Ptr} e, int att\-Num)
\begin{CompactList}\small\item\em Returns 1 if the specified attribute is continuous and 0 otherwise. \item\end{CompactList}\item 
int {\bf Example\-Get\-Discrete\-Attribute\-Value} ({\bf Example\-Ptr} e, int att\-Num)
\begin{CompactList}\small\item\em Returns the value of the indicated discrete attribute. \item\end{CompactList}\item 
double {\bf Example\-Get\-Continuous\-Attribute\-Value} ({\bf Example\-Ptr} e, int att\-Num)
\begin{CompactList}\small\item\em Returns the value of the indicated continuous attribute. \item\end{CompactList}\item 
int {\bf Example\-Is\-Class\-Unknown} ({\bf Example\-Ptr} e)
\begin{CompactList}\small\item\em Returns 1 if the value of the example's class is known, and 0 otherwise. \item\end{CompactList}\item 
float {\bf Example\-Distance} ({\bf Example\-Ptr} e, {\bf Example\-Ptr} dst)
\begin{CompactList}\small\item\em Returns the euclidian distance between the two examples. \item\end{CompactList}\item 
void {\bf Example\-Write} ({\bf Example\-Ptr} e, FILE $\ast$out)
\begin{CompactList}\small\item\em Writes the example to the passed file point.er. \item\end{CompactList}\end{CompactItemize}


\subsection{Define Documentation}
\index{Example.h@{Example.h}!ExampleGetClass@{ExampleGetClass}}
\index{ExampleGetClass@{ExampleGetClass}!Example.h@{Example.h}}
\subsubsection{\setlength{\rightskip}{0pt plus 5cm}\#define Example\-Get\-Class(e)\ ((({\bf Example\-Ptr})e) $\rightarrow$ myclass)}\label{Example_8h_a3}


Returns the value of the example's class. 

If the value of the example's class is known this returns the value, otherwise this returns -1. \index{Example.h@{Example.h}!ExampleGetNumAttributes@{ExampleGetNumAttributes}}
\index{ExampleGetNumAttributes@{ExampleGetNumAttributes}!Example.h@{Example.h}}
\subsubsection{\setlength{\rightskip}{0pt plus 5cm}\#define Example\-Get\-Num\-Attributes(e)\ (VALLength((({\bf Example\-Ptr})e) $\rightarrow$ attributes))}\label{Example_8h_a1}


Returns the number of attributes that this example has. 

This will be equal to the number of attributes that were in the Example\-Spec used to construct the example. \index{Example.h@{Example.h}!ExampleIsAttributeUnknown@{ExampleIsAttributeUnknown}}
\index{ExampleIsAttributeUnknown@{ExampleIsAttributeUnknown}!Example.h@{Example.h}}
\subsubsection{\setlength{\rightskip}{0pt plus 5cm}\#define Example\-Is\-Attribute\-Unknown(e, att\-Num)\ ( VALIndex(e $\rightarrow$ attributes, att\-Num) == 0 )}\label{Example_8h_a0}


Returns 1 if the specified attribute is marked as unknown and 0 otherwise. 

Be sure not to ask for an att\-Num $>$= {\bf Example\-Get\-Num\-Attributes(e)}. 

\subsection{Typedef Documentation}
\index{Example.h@{Example.h}!Example@{Example}}
\index{Example@{Example}!Example.h@{Example.h}}
\subsubsection{\setlength{\rightskip}{0pt plus 5cm}typedef struct {\bf \_\-Example\_\-}  {\bf Example}}\label{Example_8h_a4}


ADT for working with examples. 

See {\bf Example.h} for more detail. \index{Example.h@{Example.h}!ExamplePtr@{ExamplePtr}}
\index{ExamplePtr@{ExamplePtr}!Example.h@{Example.h}}
\subsubsection{\setlength{\rightskip}{0pt plus 5cm}typedef struct {\bf \_\-Example\_\-} $\ast$ {\bf Example\-Ptr}}\label{Example_8h_a5}


ADT for working with examples. 

See {\bf Example.h} for more detail. 

\subsection{Function Documentation}
\index{Example.h@{Example.h}!ExampleAddNoise@{ExampleAddNoise}}
\index{ExampleAddNoise@{ExampleAddNoise}!Example.h@{Example.h}}
\subsubsection{\setlength{\rightskip}{0pt plus 5cm}void Example\-Add\-Noise ({\bf Example\-Ptr} {\em e}, float {\em p}, int {\em do\-Class}, int {\em attrib})}\label{Example_8h_a17}


Randomly corrupts the attributes and class of the example. 

p should be a number between 0-1, which is interpreted as a probability (e.g. a value of .732 would be interpreted as 73.2\%). class and attrib are flags which should be 1 if you want noise added to that part of the example and 0 otherwise. Then, for each discrete thing selected by the flags, this function will have the specified probability of changing it, without replacement, to a randomly selected value. This function changes the value of each continuous attribute by adding to it a value drawn from a normal distribution with mean 0 and with standard deviation p. \index{Example.h@{Example.h}!ExampleClone@{ExampleClone}}
\index{ExampleClone@{ExampleClone}!Example.h@{Example.h}}
\subsubsection{\setlength{\rightskip}{0pt plus 5cm}{\bf Example\-Ptr} Example\-Clone ({\bf Example\-Ptr} {\em e})}\label{Example_8h_a11}


Allocates memory and copies the example into it. 

\index{Example.h@{Example.h}!ExampleDistance@{ExampleDistance}}
\index{ExampleDistance@{ExampleDistance}!Example.h@{Example.h}}
\subsubsection{\setlength{\rightskip}{0pt plus 5cm}float Example\-Distance ({\bf Example\-Ptr} {\em e}, {\bf Example\-Ptr} {\em dst})}\label{Example_8h_a25}


Returns the euclidian distance between the two examples. 

This ignores discrete attributes. \index{Example.h@{Example.h}!ExampleFree@{ExampleFree}}
\index{ExampleFree@{ExampleFree}!Example.h@{Example.h}}
\subsubsection{\setlength{\rightskip}{0pt plus 5cm}void Example\-Free ({\bf Example\-Ptr} {\em e})}\label{Example_8h_a10}


Frees all the memory being used by the passed example. 

\index{Example.h@{Example.h}!ExampleGetContinuousAttributeValue@{ExampleGetContinuousAttributeValue}}
\index{ExampleGetContinuousAttributeValue@{ExampleGetContinuousAttributeValue}!Example.h@{Example.h}}
\subsubsection{\setlength{\rightskip}{0pt plus 5cm}double Example\-Get\-Continuous\-Attribute\-Value ({\bf Example\-Ptr} {\em e}, int {\em att\-Num})}\label{Example_8h_a23}


Returns the value of the indicated continuous attribute. 

If the att\-Num is valid, and Example\-Get\-Attribute\-Unknown(e, att\-Num) returns 0, and Example\-Is\-Attribute\-Continuous(e, att\-Num) returns 1, this function will return the value of the attribute. If the conditions aren't met, there is a good chance that calling this will crash your program. \index{Example.h@{Example.h}!ExampleGetDiscreteAttributeValue@{ExampleGetDiscreteAttributeValue}}
\index{ExampleGetDiscreteAttributeValue@{ExampleGetDiscreteAttributeValue}!Example.h@{Example.h}}
\subsubsection{\setlength{\rightskip}{0pt plus 5cm}int Example\-Get\-Discrete\-Attribute\-Value ({\bf Example\-Ptr} {\em e}, int {\em att\-Num})}\label{Example_8h_a22}


Returns the value of the indicated discrete attribute. 

If the att\-Num is valid, and Example\-Get\-Attribute\-Unknown(e, att\-Num) returns 0, and Example\-Is\-Attribute\-Discrete(e, att\-Num) returns 1, this function will return the value of the attribute. If the conditions aren't met, there is a good chance that calling this will crash your program. \index{Example.h@{Example.h}!ExampleIsAttributeContinuous@{ExampleIsAttributeContinuous}}
\index{ExampleIsAttributeContinuous@{ExampleIsAttributeContinuous}!Example.h@{Example.h}}
\subsubsection{\setlength{\rightskip}{0pt plus 5cm}int Example\-Is\-Attribute\-Continuous ({\bf Example\-Ptr} {\em e}, int {\em att\-Num})}\label{Example_8h_a21}


Returns 1 if the specified attribute is continuous and 0 otherwise. 

Be sure not to ask for an att\-Num $>$= {\bf Example\-Get\-Num\-Attributes(e)}. \index{Example.h@{Example.h}!ExampleIsAttributeDiscrete@{ExampleIsAttributeDiscrete}}
\index{ExampleIsAttributeDiscrete@{ExampleIsAttributeDiscrete}!Example.h@{Example.h}}
\subsubsection{\setlength{\rightskip}{0pt plus 5cm}int Example\-Is\-Attribute\-Discrete ({\bf Example\-Ptr} {\em e}, int {\em att\-Num})}\label{Example_8h_a20}


Returns 1 if the specified attribute is discrete and 0 otherwise. 

Be sure not to ask for an att\-Num $>$= {\bf Example\-Get\-Num\-Attributes(e)}. \index{Example.h@{Example.h}!ExampleIsClassUnknown@{ExampleIsClassUnknown}}
\index{ExampleIsClassUnknown@{ExampleIsClassUnknown}!Example.h@{Example.h}}
\subsubsection{\setlength{\rightskip}{0pt plus 5cm}int Example\-Is\-Class\-Unknown ({\bf Example\-Ptr} {\em e})}\label{Example_8h_a24}


Returns 1 if the value of the example's class is known, and 0 otherwise. 

\index{Example.h@{Example.h}!ExampleNew@{ExampleNew}}
\index{ExampleNew@{ExampleNew}!Example.h@{Example.h}}
\subsubsection{\setlength{\rightskip}{0pt plus 5cm}{\bf Example\-Ptr} Example\-New ({\bf Example\-Spec\-Ptr} {\em es})}\label{Example_8h_a8}


Programmatically creates a new example. 

Allocates enough memory to hold all the attributes mentioned in the Example\-Spec passed. Use the Example\-Set\-Foo functions (see below) to set the values of the attributes and class.

This function allocates memory which should be freed by calling Example\-Free. \index{Example.h@{Example.h}!ExampleRead@{ExampleRead}}
\index{ExampleRead@{ExampleRead}!Example.h@{Example.h}}
\subsubsection{\setlength{\rightskip}{0pt plus 5cm}{\bf Example\-Ptr} Example\-Read (FILE $\ast$ {\em file}, {\bf Example\-Spec\-Ptr} {\em es})}\label{Example_8h_a18}


Attempts to read an example from the passed file pointer. 

FILE $\ast$ should be opened for reading. The file should contain examples in C4.5 format. Uses the Example\-Spec to determine how many, what types, and how to interpret the attributes it needs to read.

This function will return 0 (NULL) if it is unable to read an example from the file (bad input or EOF). If the input is badly formed, the function will also output an error to the console.

Note that you could pass STDIN to the function to read an example from the console.

This function allocates memory which should be freed by calling Example\-Free. \index{Example.h@{Example.h}!ExampleSetAttributeUnknown@{ExampleSetAttributeUnknown}}
\index{ExampleSetAttributeUnknown@{ExampleSetAttributeUnknown}!Example.h@{Example.h}}
\subsubsection{\setlength{\rightskip}{0pt plus 5cm}void Example\-Set\-Attribute\-Unknown ({\bf Example\-Ptr} {\em e}, int {\em att\-Num})}\label{Example_8h_a12}


Marks the specified attribute value as unknown. 

Future calls to Example\-Get\-Attribute\-Unknown for that attribute will return 1. \index{Example.h@{Example.h}!ExampleSetClass@{ExampleSetClass}}
\index{ExampleSetClass@{ExampleSetClass}!Example.h@{Example.h}}
\subsubsection{\setlength{\rightskip}{0pt plus 5cm}void Example\-Set\-Class ({\bf Example\-Ptr} {\em e}, int {\em the\-Class})}\label{Example_8h_a15}


Sets the example's class to the specified class. 

The function doesn't do much error checking so be sure that you call it consistent with Example\-Spec\-Get\-Num\-Classes. If you don't, there is a chance the example could cause your program to crash. \index{Example.h@{Example.h}!ExampleSetClassUnknown@{ExampleSetClassUnknown}}
\index{ExampleSetClassUnknown@{ExampleSetClassUnknown}!Example.h@{Example.h}}
\subsubsection{\setlength{\rightskip}{0pt plus 5cm}void Example\-Set\-Class\-Unknown ({\bf Example\-Ptr} {\em e})}\label{Example_8h_a16}


Marks the example's class as unknown. 

\index{Example.h@{Example.h}!ExampleSetContinuousAttributeValue@{ExampleSetContinuousAttributeValue}}
\index{ExampleSetContinuousAttributeValue@{ExampleSetContinuousAttributeValue}!Example.h@{Example.h}}
\subsubsection{\setlength{\rightskip}{0pt plus 5cm}void Example\-Set\-Continuous\-Attribute\-Value ({\bf Example\-Ptr} {\em e}, int {\em att\-Num}, double {\em value})}\label{Example_8h_a14}


Considers the specified attribute to be continuous and sets its value to the specified value. 

This function doesn't do much error checking so be sure that you call it consistent with Example\-Is\-Attribute\-Discrete, and Example\-Is\-Attribute\-Continuous. If you don't, there is a chance the example could cause your program to crash. \index{Example.h@{Example.h}!ExampleSetDiscreteAttributeValue@{ExampleSetDiscreteAttributeValue}}
\index{ExampleSetDiscreteAttributeValue@{ExampleSetDiscreteAttributeValue}!Example.h@{Example.h}}
\subsubsection{\setlength{\rightskip}{0pt plus 5cm}void Example\-Set\-Discrete\-Attribute\-Value ({\bf Example\-Ptr} {\em e}, int {\em att\-Num}, int {\em value})}\label{Example_8h_a13}


Considers the specified attribute to be discrete and sets its value to the specified value. 

This function doesn't do much error checking so be sure that you call it consistent with Example\-Is\-Attribute\-Discrete, Example\-Is\-Attribute\-Continuous and Example\-Spec\-Get\-Attribute\-Value\-Count. If you don't, there is a chance the example could cause your program to crash. \index{Example.h@{Example.h}!ExamplesRead@{ExamplesRead}}
\index{ExamplesRead@{ExamplesRead}!Example.h@{Example.h}}
\subsubsection{\setlength{\rightskip}{0pt plus 5cm}{\bf Void\-List\-Ptr} Examples\-Read (FILE $\ast$ {\em file}, {\bf Example\-Spec\-Ptr} {\em es})}\label{Example_8h_a19}


Reads as many examples as possible from the file pointer. 

Calls Example\-Read until it gets a 0, allocates a list and adds each example to it. You are responsible for freeing the examples and the list. \index{Example.h@{Example.h}!ExampleWrite@{ExampleWrite}}
\index{ExampleWrite@{ExampleWrite}!Example.h@{Example.h}}
\subsubsection{\setlength{\rightskip}{0pt plus 5cm}void Example\-Write ({\bf Example\-Ptr} {\em e}, FILE $\ast$ {\em out})}\label{Example_8h_a26}


Writes the example to the passed file point.er. 

FILE $\ast$ should be opened for writing. The example will be written in C4.5 format, and could later be read in using Example\-Read.

Note that you could pass stdout to the function to write an example to the console. 