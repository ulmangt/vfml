\section{REPrune.h File Reference}
\label{REPrune_8h}\index{REPrune.h@{REPrune.h}}


\subsection{Detailed Description}
Peforms reduced error pruning on a decision tree. 

Takes a decision tree and a collection of examples and prunes every portion of the tree that does not contribute to an increase of accuracy on the data set. This requires a single pass over the prune set, and then on order of one pass over the tree for each node that is pruned.

REPrune uses the growing data fields of the decision tree, so it will overwrite anything that you may have there. If you want to trying pruning a tree and then continue growing it you should make a clone of the tree and pass the clone to this interface so that the origional will remain untouched.

\begin{Desc}
\item[{\bf Wish List}]More pruning methods, and more sophisticated REPruning (for example, all the leaves to prune could be selected in a single pass over the tree). \end{Desc}


\subsection*{Functions}
\begin{CompactItemize}
\item 
void {\bf REPrune\-Batch} ({\bf Decision\-Tree\-Ptr} dt, Void\-AList\-Ptr examples)
\begin{CompactList}\small\item\em Get the prune set from the in-memory list. \item\end{CompactList}\item 
void {\bf REPrune\-Batch\-File} ({\bf Decision\-Tree\-Ptr} dt, FILE $\ast$example\-In, long prune\-Max)
\begin{CompactList}\small\item\em Get the prune set from the file. \item\end{CompactList}\end{CompactItemize}


\subsection{Function Documentation}
\index{REPrune.h@{REPrune.h}!REPruneBatch@{REPruneBatch}}
\index{REPruneBatch@{REPruneBatch}!REPrune.h@{REPrune.h}}
\subsubsection{\setlength{\rightskip}{0pt plus 5cm}void REPrune\-Batch ({\bf Decision\-Tree\-Ptr} {\em dt}, Void\-AList\-Ptr {\em examples})}\label{REPrune_8h_a0}


Get the prune set from the in-memory list. 

\index{REPrune.h@{REPrune.h}!REPruneBatchFile@{REPruneBatchFile}}
\index{REPruneBatchFile@{REPruneBatchFile}!REPrune.h@{REPrune.h}}
\subsubsection{\setlength{\rightskip}{0pt plus 5cm}void REPrune\-Batch\-File ({\bf Decision\-Tree\-Ptr} {\em dt}, FILE $\ast$ {\em example\-In}, long {\em prune\-Max})}\label{REPrune_8h_a1}


Get the prune set from the file. 

Do not use more than prune\-Max of the examples from the file for pruning; setting prune\-Max to 0 means 'no max' (and uses all examples from the file for pruning). 